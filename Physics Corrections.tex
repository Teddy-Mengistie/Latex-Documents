documentclass[12pt,letterpaper]{article}
usepackage[utf8]{inputenc}
usepackage{amsmath}
usepackage{graphicx}
usepackage{tikz}
title{Physics Quiz Corrections}
author{Teddy M}
date{November 4, 2019}
pagestyle{plain}
begin{document}
maketitle
begin{flushleft}
   setcounter{secnumdepth}{0}
section{Question 2}
     A ball is thrown from a point 24 m above the ground, strikes the ground after traveling horizontally a distance of 18 m. With what speed was it thrown?
	subsection{Original Answer}
	I chose E) Can not be determined with the data provided.
	subsection{How I got my answer}
	I got my answer because on the test I misinterpreted the question by thinking that the motion was only up and down for the ball.
	
	subsection{Correct Answer}
	The correct answer was C) $8.1m/s$.
	subsubsection{Math}
We were given that:
begin{tabular}{||c c||} 
 hline
 emph{x} & emph{y} \ [0.5ex] 
 hlinehline
$Delta d = 18m$ & $d_{i} = 24m$ \ 
 hline
  & $a = 9.8 m/s^{2}$\
 hline
  & $V_{i} = 0$ \
 hline
end{tabular}\
The first step was determining the time because that is when emph{x} and emph{y} both corroborate. I did this by using one of the most important equations.\
begin{center}
$Delta d = V_{i} t + frac{1}{2}at^{2} $\
$24 = frac{1}{2}(9.8)(t^{2})$\
$frac{24}{4.9} = frac{4.9t^{2}}{4.9}$\
$sqrt{t^{2}}=sqrt{4.89}$\
$t= 2.21$
end{center}
When I plugged in the values for the equation, I found out that  $t=2.21s$. I then needed to know the $V_{x}$ because the ball does not have a vertical velocity at the start. Then 
I divided the $d_{x}$(18m) by the time which is 2.21s because I wanted to find the initial velocity in the x direction. I found and answer that was approximately 8.1 m/s.
	section{Question 10}
	A ball is thrown and follows the parabolic path shown above. Air friction is negligible. Point Q is the highest point on the path. Points P and R are the same height above the ground.
	setcounter{secnumdepth}{0}
	subsection{Original Answer}
	I chose  B)emph{"$V_R<V_Q<V_P$"} as my answer for question 10.
	subsection{How I got my answer}
	I chose my answer because I thought that the question was talking about emph{speed} and I thought of the emph{speed} as a emph{velocity} and I thought that point $V_{R}$ had a negative velocity. My answer would have been correct if the question was talking about speed, then direction would not matter because speed is a scalar.
    subsection{Correct Answer}
  	D is the correct answer because $V_R$ and $V_P$ have the same speed because it was indicated in the question that they are both at the same height.
section{Question 12}
	An object is released from rest and falls a distance emph{h} during the first second of time. How far will it fall during the next second of time?
	subsection{Original Answer}
	I answered D)4emph{h}.
	subsection{How I got my answer}
	I got my answer because I calculated the $d_{f}$ for the problem but never subtracted the $d_{i}$ which was emph{h} to find the displacement.
	subsection{Correct Answer}
	The correct answer for this problem is C)3emph{h}.
	subsubsection{Math}
begin{small}
In the problem, we were given:\
begin{center}
begin{tabular}{||c||} 
 hline
 emph{y} \ [0.5ex] 
 hline
 hline
$d_{i} = 24m$ \ 
 hline
$a = 9.8 m/s^{2}$\
 hline
$V_{i} = 0$ \
 hline
$t = 2s$ \
hline
end{tabular}\
end{center}
I used the equation $Delta d = V_{i} t + frac{1}{2}at^{2}$ to find out what emph{h} was equal to. \
begin{center}
$h=0t+frac{1}{2}(9.8)1^{2}$\
$h=frac{1}{2}(9.8)$\
$h=4.9m$\ 
end{center}

After I found h, I found the $d_{f}$ so that I could find the displacement($ d_{f}-h $) when the object is 2 seconds into free fall.I found the $d_{f}$ by doing:end{small}\
begin{center}
$d_{f}=frac{1}{2}(9.8)(2^{2}$\
$d_{f}=9.8cdot2$\
$d_{f}= (hcdot2)cdot2$\
$d_{f}=4h$\
$Delta d = 4h - h$\
$Delta d = 3h$\
end{center}
begin{center}
{huge Free Response Questionspar}
end{center}
    
    section{Question 13}
	subsection{13a}
	subsubsection{Original Answer}
	I originally found that the acceleration was $2.5m/s^{2}$
	subsubsection{How I got my answer}
	I found my answer by setting the a(acceleration) to $frac{V}{t}$ instead of $frac{Delta V}{Delta t}$.
	subsubsection{Correct Answer}
	$a=5m/s^{2}$\
	Math\
	I found the right answer by substituting my values of the given into the equation:
	begin{center}
	 $Delta d = V_{i} t + frac{1}{2}at^{2}$\
	 $frac{10}=frac{frac{1}{2}a(4)}{4}$\
	 $(2)frac{10}{4}=frac{1}{2}a(2)$\
	 $a=frac{20}{4}$\
	 $a = 5 m/s^{2}$
	end{center}	 
	subsection{13b}
	Determine the sprinter's velocity after 2 seconds have elapsed.
	subsubsection{Original Answer}
	$5m/s$.
	subsubsection{How I got my answer}
	I got my answer because I used the acceleration that I found on 13a on this question.
	subsubsection{Correct Answer}
	$10m/s$\
	I found the $V_{f}$ by using this equation:\
	begin{center}
	$a=frac{Delta V}{Delta t}$\
	$5=frac{V_{f}-V_{i}}{2}$\
	$5=frac{V_{f}}{2}$\
	$V_{f}=5cdot{2}$\
	$V_{f}=10m/s^{2}$
	end{center}
	subsection{13c}
	On the axes provided below, draw the displacement vs. time curve for the sprinter.
	subsubsection{Original Answer}
	20 seconds.
	subsubsection{How I got my answer}
	I got my answer by using the wrong velocities and accelerations to find when the person completes the 100m. 
	subsubsection{Correct Answer}
	11 seconds.I found this answer by finding the time it takes to run the other 90 meters that the person has not ran in the 2 seconds.\
	Math\
	begin{center}
	$Delta d = V_{i} t + frac{1}{2}at^{2}$\
	$90=10t+frac{1}{2}0t^{2}$\
	$90=10t$\
	$t=90/10=9$\
	$t=9s$\
	$t_{total}=9+2$\
	$t_{total}=11s$
	end{center}
section{Question 15b}
	subsection{Original Answer}
	I had not done anything but written the givens of the problem for 15b.\
	begin{center}
begin{tabular}{||c c||} 
 hline
 emph{x} & emph{y} \ [0.5ex] 
 hline
 hline
$V_{ix} = 3.58m/s$ & $d_{yi} = 2.44m$ \ 
 hline
$d_{x} = 4.57m$ & $a = -9.8 m/s^{2}$\
 hline
$a=0$&$V_{iy} = 6.2 m/s$ \
 hline
end{tabular}\
end{center}
	I wrote that the person making the shot made the basket ball in the hoop.
	subsection{How I got my answer}
	I got my answer because a 60 degree angle shot seemed like a reasonable angle to shoot at from that close of a distance, so I assumed that he would make the shot. I interpreted the question this way because I did not have enough time to complete the last question.
	subsection{Correct Answer}
	This was not the case, the person did not make the shot nor was he that close. 
	subsubsection{Math}
	Finding the correct answer took two steps, first I had to figure out the time at where the ball is as far to the shooter as the ball on the x direction. Then I had to plug the time into the y equation, and then solve for the distance of the ball at that time.\
	begin{center}
	 $Delta d_{x} = V_{ix} t + frac{1}{2}at^{2}$\
	 $4.57=3.58t+frac{1}{2}at^{2}$\
	 $4.57=3.58t$\
	 $t=frac{4.57}{3.58}$\
	$t=1.28s$\
	 $x = x_{0}+V_{ix} t + frac{1}{2}at^{2}$\
	 $x =2.44+6.2(1.28) + (-4.9)(1.28)^{2}$\
	 $ x= 2.44+7.9851-8.15409$\
	 $ x=2.44+.160899$\
	 $xapprox 2.6m$\
	 The height of the ball at the time it has an equal distance from the shooter as the hoop is too low to make the shot. He missed the shot by 0.45 meters. Which  is a lot because it is in meters.
	end{center}
end{flushleft}
end{document}